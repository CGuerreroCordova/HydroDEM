\chapter*{Abstract}
\label{chap:abstract}

In this work a methodology for the adaptation of a Digital Elevation Model (DEM) for hydrological uses was developed.

"`Adaptation"' or "`hydrological conditioning"' refers to the \textit {Modification of the height data of a DEM to more accurately represent the flow of water through the surface.}

The processes corresponding to this methodology were automated using a processing chain developed in Python: Tree curtains are eliminated, high frequency banding effect is corrected, lagoon detection algorithm is developed, terrain values ​​are obtained in areas corresponding to gullies and finally these processes are combined in the DEM that is obtained as a result of this processing.

Detailed description of the methodology developed is provided as a result of this work, in addition the software responsible for carrying out these processes. This program can be executed by passing as arguments the input data necessary for processing.

The methodology is based on a previous work (Masuelli 2009 \textit {et. al} \cite{Masuelli2009}) in which tasks of conditioning of the DEM were performed manually. This work was used as a baseline, (with some variations and modifications) and with the added value of carrying out most of the processing automatically, in addition several improvements are made and finally a validation of the processed DEM is made with a 2D hydrological model.

The southern part of the province of Córdoba is chosen as area of ​​interest for the development of this work, since it is an area that in recent years has presented numerous periods of flooding of the plain. This type of flooding is characterized by the presence of numerous scattered lagoons.

Statistically results are analyzed, finding a substantial improvement of the processed DEM with respect to the DEM HydroSHEDS.

\textbf{Keywords: DEM, Hydrology, water, filter, pixel, processing, mask, flow, heights}