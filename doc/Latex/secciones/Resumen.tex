\chapter*{Resumen}
\label{chap:resumen}


En este trabajo se desarrolló una metodología para la adecuación de un Modelo Digital de Elevación (DEM) para usos hidrológicos. Se entiende por "`adecuación"' o "`acondicionamiento"' hidrológico o para usos hidrológicos a la \textit{Modificación de los datos de altura de un DEM para representar más precisamente el movimiento del agua a través de la superficie.}

Los procesos correspondientes a dicha metodología fueron automatizados mediante una cadena de procesamiento desarrollado en Python: Se eliminan cortinas de árboles, se corrige un efecto de bandeado de alta frecuencia, se desarrolla un algoritmo de detección de lagunas, se obtienen valores del terreno correspondientes a zonas de cañadas y finalmente se combinan estos procesamientos en el DEM que se obtiene como resultado de este procesamiento.

De esta manera, como resultado de este trabajo, se provee la descripción detallada de la metodología desarrollada, además del programa encargado de realizar estos procesamientos, dicho programa se puede ejecutar pasándole como argumentos los datos de entrada necesarios para el procesamiento.

La metodología toma como base un trabajo previo (Masuelli 2009 \textit{et. al} \cite{Masuelli2009}), en el cual se realizaron tareas de adecuación del DEM de forma manual. En esta ocasión se sigue la línea de dicho trabajo (con algunas variantes y modificaciones), con el valor agregado de realizar la mayor parte del procesamiento de manera automática, además de aportar varias mejoras. Finalmente, se realiza una validación del DEM procesado con un modelo hidrológico 2D.

Se elige como área de interés para el desarrollo de este trabajo el sur de la provincia de Córdoba, ya que es una zona que en los últimos años ha presentado numerosos períodos de inundaciones de llanura. Este tipo de inundaciones se caracteriza por contar con la presencia de numerosas lagunas dispersas.

Se analizan resultados estadísticamente encontrándonse una mejora sustancial del DEM procesado respecto del DEM HydroSHEDS.


\textbf{Palabras clave: DEM, Hidrología, agua, filtro, píxel, procesamiento, máscara, flujo, alturas}